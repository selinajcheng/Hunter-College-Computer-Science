\documentclass[12pt, letterpaper]{article}
\usepackage[margin = 1in]{geometry} % sets 1-inch margin on all sides

%- Language and encoding
\usepackage[T1]{fontenc}
\usepackage[utf8]{inputenc}
\usepackage[english]{babel}
\usepackage{microtype}

%- Bibliography-related
\usepackage{cite}
\bibliographystyle{IEEEtran}

\usepackage{graphicx}
\usepackage{booktabs}
\usepackage{fancyhdr}
\usepackage{csquotes}

\usepackage{hyperref}
\usepackage[nameinlink]{cleveref}

\usepackage{titling}
\usepackage{parskip}
\usepackage{etoolbox}

%- Fonts and headings
\usepackage{helvet}
\usepackage{titlesec}
% Set all headings to sans serif font
\titleformat{\section}{\fontfamily{lmss}\selectfont\Large\bfseries}{}{}{}[]
\titleformat{\section}[runin]
  {\sffamily\bfseries\medium} % font: sans-serif bold large
  {}{0em}{}[\quad] % 1em space after label; \quad space after title
\titleformat{\subsection}{\fontfamily{lmss}\selectfont\medium\bfseries}{}{}{}[]
\titleformat{\subsubsection}{\fontfamily{lmss}\selectfont\normalsize\bfseries}{}{}{}[]

%— Paragraph spacing
\setlength{\parskip}{.5ex plus .2ex minus .2ex}
\setlength{\parindent}{0pt}

\begin{document}
\pagestyle{empty}
\singlespacing
\vspace{1.0cm}

% full bibliographic reference including title, author, date, venue. 
\section{Subject:}
\label{sec:subject}
Y. Huang, L. Liang, T. Li, X. Jia, R. Wang, W. Miao, G. Pu, and Y. Link, "Perception-guided jailbreak against text-to-image models,” 2025. [Online]. Available: https://arxiv.org/abs/2408.10848

% describe the content, including its three most important points.
\section{Summary:}
\label{sec:summary}
The authors of this paper propose a model-free and model-agnostic black-box jailbreak method that generates adversarial attack prompts: perception-guided jailbreak (PGJ). The researchers were guided by the goals of \textbf{a)} avoiding nonsensical tokens to increase stealthiness, \textbf{b)} a simple pipeline to decrease time and resource consumption and \textbf{c)} achieving similar human perception. \\

The authors term perceptual confusion as when visual similarity of elements in a scene are confused by people for different objects or behaviors. The method leverages this concept to automatically find safe substitution phrases evaluated by what the authors name the Perception Similarity and Text Semantic Inconsistency (PSTSI) principle, which states that the safe substitution phrases and the target unsafe word should be perceived similarly by humans but inconsistent semantically in text.\\

Computationally representing human perception is challenging but the authors leverage LLMs' ability to detect unsafe words and understanding of real-world visual properties to identify unsafe words to replace and their corresponding safe substitution phrases. This method focuses on bypassing the pre-checker of safety checkers which check the textual input before generation. 

% Identify the author’s underlying thesis in a single sentence.
\section{Thesis Statement:}
\label{sec:thesis}
PGJ can be effectively used to bypass text-to-image (T2I) model safety checkers without targeting post-checkers, revealing the vulnerability of the current safety mechanisms of T2I models.

% Identify the principal results of the research. (If it is a survey paper, indicate instead the major points instead.)
\section{Principal Results:}
\label{sec:principal}
The PGJ method was tested across 5 NSFW categories using 1,000 prompts generated by GPT-4 on six open-source and commercial T2I models: DALLE-2, DALLE-3, Cogview3, SDXL (open-source), Tongyiwanxiang, and Hunyuan. The PGJ method focuses on targeting the pre-checkers to bypass text filters, which poses a greater challenge than image filters. Baseline methods (SneakyPrompt, MMA-Diffusion, DACA, Ring-A-Bell) were chosen and evaluated for performance on the same prompts as PGJ. It was found that the success rate was 0.915 on average and consistent across the tested T2I models. On average, PGJ took 5.51 seconds to modify unsafe prompts. PGJ was found to significantly outperform state-of-the-art attack methods both in effectiveness and efficiency.

% \pagebreak

% What parts were the clearest? What parts were the least clear?
\section{Clarity:}
\label{sec:clarity}
The PSTSI principle formula was hard to understand at first glance, but the example helped me understand it better. The authors were extensive and clear in describing their methods, motivation, and the relevant background information needed to understand their work in depth.

% How does the paper relate to others you have read? Provide full bibliographic citations. At least say what kind of papers would have been relevant. If you reference one of their references, say something specific about it
\section{Relevance:}
\label{sec:relevance}
The PGJ method doesn't rely on a model and solely uses LLMs to operate. The way the authors optimize for this relates to the findings on prompting with examples found in the Reid et al. 2025 paper \cite{reid2025doesgptreallyit}. Chain-of-thought reasoning and examples were found to be beneficial, just as it is here to achieve visually similar safe words or description. This paper is also similar to a referenced method called ColJailBreak, by Ma et al. \cite{NEURIPS2024_6f11132f}. ColJailBreak also takes advantage of human perception, namely shapes, to generate unsafe images.

% Ask at least three thoughtful questions, now that you have read it. These might be thesis topic ideas, or ways the material relates to other work in problem solving or statistics or computer science with which you are familiar.
\section{Further Study:}
\label{sec:further}
How can LLMs guard against methods that use \emph{itself} to generate unsafe prompts? Is it possible for LLMs to guard against instructions like \emph{Instruction 2} as used in the paper, or would that interfere would product usability? Does the pre- or post-checker pose a greater obstacle in successful unsafe generation? Are there LLMs that differ from the victim T2I models PGJ may work less (or even more) effectively on? Would PGJ work even more effectively on GPT-4? 

% Why would you reread this paper, or recommend it to a colleague? This should not be a review.
\section{Recommendation:}
\label{sec:rec}
I would recommend this paper to anyone who engages with LLM models, have an interest in cybersecurity, and/or jailbreaking. PGJ is intriguing because of its lack of need for a model and reliance on LLMs (GPT-4) to fool other LLMs.

% What impact the methods, data, and/or results will have on the community beyond computer science.
\section{Social Impact:}
\label{sec:impact}
This has implications for any edtech services or educational institutions that use or engage with LLMs. Examples of implications include misuse of services and violations of child safety if unsafe content can be generated. Children are also more likely to be sensitive to perceptual confusion, which makes PGJ or similar adversarial attacks concerning. Implications also exist in content moderation and filtering, e.g., in search results and social media.

% This can be notation, methods, language, or ideas. A list is not enough, however. Include at least 3 references to demonstrate that you did look them up.
\section{What I looked up:}
\label{sec:looked up}
PSTSI principle (before I saw it defined later in the research paper), black-box vs white-box jailbreak, \cite{gfg_black_white_testing}, the code base the authors created to implement PGJ \cite{huang2025pgjcode}

\bibliography{references}

\end{document}